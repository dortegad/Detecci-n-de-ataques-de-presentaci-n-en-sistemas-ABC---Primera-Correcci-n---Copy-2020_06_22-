\newglossaryentry{ABC}
{
    name        = {ABC},
    plural      = {ABCs},
    description = {\textit{Automatic Border Control}.}
}

\newglossaryentry{ABC4EU}
{
    name        = {ABC4EU},
    text        = \mbox{ABC4EU},
    description = {Proyecto europeo \GLS{FP7} dedicado a la investigación en el campo de los sistemas \GLS{ABC} \cite{ABC4EUOnline}.}
}

\newglossaryentry{ACC}
{
    name        = {ACC},
    description = {Ratio de la precisión de un sistema.}
}

\newglossaryentry{ACER}
{
    name        = {ACER},
    description = {\textit{Attack Clasificated Error Rate}.}
}

\newglossaryentry{AFIS-ABIS}
{
    name        = {AFIS-ABIS},
    text        = {\mbox{AFIS-ABIS}},
    description = {\textit{Automated Biometric Fingerprint Identification System}. Sistema encargado del almacenamiento, la edición, y la búsqueda de los datos biométricos de los viajeros en los sistemas \GLS{ABC}.}
}

\newglossaryentry{AFIS}
{
    name        = {AFIS},
    description = {\textit{Automated Fingerprint Identification System}. Sistema para la identificación de huellas dactilares mantenida por el FBI.}
}

\newglossaryentry{AGAST}
{
    name        = {AGAST},
    description = {\textit{Adaptive and Generic Accelerated Segment Test}. Algoritmo de detección y descripción de puntos de interés en una imagen. Basado en FAST y descrito en el estudio \cite{mair2010_agast}.}
}

\newglossaryentry{AIT}
{
    name        = {AIT},
    description = {\textit{Austrian Institute of Technology}. Institución encargada del proyecto Modentity \cite{ModentityOnline}.}
}

\newglossaryentry{antispoofing}
{
    name        = {antispoofing},
    text        = {\textit{antispoofing}},
    description = {Métodos \textit{software} o \textit{hardware} para evitar los ataques de presentación (\GLS{PA}) o \textit{\gls{spoofing}}.}
}

\newglossaryentry{AlexNet}
{
    name        = {AlexNet},
    description = {Arquitectura \GLS{CNN} para visión artificial presentada en la publicación \cite{krizhevsky2012imagenet}.}
}

\newglossaryentry{APCER}
{
    name        = {APCER},
    description = {\textit{Attack Presentation Classification Error Rate}. Ataques de presentación considerados \textit{bona-fide}.}
}

\newglossaryentry{ARH}
{
    name        = {ARH},
    description = {Empresa dedicada al desarrollo de sistemas ABC.}
}

\newglossaryentry{autoencoder}
{
    name        = {autoencoder},
    text        = {\textit{autoencoder}},
    description = {Arquitectura \GLS{CNN}, basada en un codificador y un decodificador, habitualmente se emplea para procesos de eliminación de ruido y de compresión.}
}

\newglossaryentry{BCP}
{
    name        = {BCP},
    description = {\textit{Border Crossing Point}. Punto de control fronterizo.}
}

\newglossaryentry{BFF}
{
    name        = BFF,
    description = {\textit{\Gls{bona-fide} Frames}. En el sistema \gls{FlyPAD},el porcentaje de fotogramas que el módulo \gls{PAD} considera \textit{bona-fide}, en un \textit{tracking valido}, para sea considerado un \textit{tracking bona-fide}.}
}

\newglossaryentry{BIOinPAD}
{
    name        = BIOinPAD,
    text        = \mbox{BIOinPAD},
    description = {\textit{BIO-Inspired Face Recognition From Multiple Viewpoints. Evauation in a Presentation Attack Detection Environment} Proyecto de investigación de la URJC dedicado a la detección de ataques con sensores bioinspirados.}
}

\newglossaryentry{biometria}
{
    name        = biometría,
    description = {Uso de rasgos físicos o de comportamiento para la identificación de individuos.}
}

\newglossaryentry{bona-fide}
{
    name        = bona-fide,
    text        = {\mbox{\textit{bona-fide}}},
    description = {Una presentación \textit{bona-fide} consiste en la presentación por parte de un individuo de sus datos biométricos reales al sistema biométrico.}
}

\newglossaryentry{BPCER}
{
    name        = BPCER,
    description = {\textit{Bona-Fide Presentation Classification Error Rate}. Presentaciones \textit{bona-fide} consideradas como ataques.}
}

\newglossaryentry{BRISQUE}
{
    name        = BRISQUE,
    description = {Índice para evaluar la calidad de las imágenes.}
}

\newglossaryentry{BSI}
{
    name        = BSI,
    description = {\textit{Bundes-Amt für Sicherheit in der Informations-Technik}. Institución dedicada a la certificación y a la creación de estándares \cite{BSIOnline}.}
}

\newglossaryentry{BSIF}
{
    name        = BSIF,
    description = {\textit{Binary Robust Independent Elementary Features}. Descriptor de imágenes basado en \GLS{SIFT}, presentado en el estudio \cite{kannala2012bsif}.}
}

\newglossaryentry{BVS}
{
    name        = BVS,
    description = {\textit{Biometric Verification System}, Subsistema de los sistemas \GLS{ABC} que se encarga de verificar la información biométrica del viajero.}
}

\newglossaryentry{captura}
{
    name        = {captura},
    description = {captura biométrica. Subsistema de un sistema biométrico que se encarga de la adquisición del rasgo biométrico a analizar.}
}

\newglossaryentry{CASIA}
{
    name        = {CASIA},
    description = {Organización con múltiples base de datos biométricas.}
}

\newglossaryentry{CCD}
{
    name        = {CCD},
    description = {\textit{charge-coupled device.}.Sensor compuesto de una matriz de cédulas fotoeléctricas que permiten registrar la imágenes en las cámaras digitales.}
}

\newglossaryentry{CEN}
{
    name        = {CEN},
    description = {\textit{European Committee for Standardization}.Organización europea para el normalizado y la estandarización en sistemas electrónicos e informáticos \cite{CENOnline}}
}

\newglossaryentry{CF}
{
    name        = CF,
    description = {\textit{Consecutive Frames}. En el sistema \gls{FlyPAD}, el número de \textit{frames} consecutivos en los que se ha detectado una identidad para considera un \text{tracking} como \textit{tracking} valido.}
}

\newglossaryentry{chip}
{
    name        = {chip},
    text        = \textit{<<chip>>},
    description = {Información biométrica del viajero almacena en los \gls{eMRTD} y que debe cumplir las especificaciones de \GLS{ICAO} \cite{ICAOOnline}.}
}

\newglossaryentry{chip bona-fide}
{
    name        = {chip bona-fide},
    text        = \textit{<<chip \mbox{bona-fide}>>},
    description = {Información biométrica del viajero almacena en los \gls{eMRTD}, que no ha sido manipulada y que presenta únicamente los rasgos biométricos del viajero genuino de la documentación.}
}

\newglossaryentry{chip morphing}
{
    name        = {chip morphing},
    text        = \textit{<<chip morphing>>},
    description = {Información biométrica del viajero que ha sido almacenada en sus \gls{eMRTD} y que ha sido manipulada mediante \gls{morphing}.}
}

\newglossaryentry{Cognitec}
{
    name        = {Cognitec},
    description = {Empresa dedicada al desarrollo de \textit{software} de reconocimiento facial \cite{cognitec2019url}.}
}

\newglossaryentry{COT}
{
    name        = {COT},
    plural      = {COTS},
    description = {\textit{Commercial Off-The-Shelf}. \textit{Software} comercial.}
}

\newglossaryentry{challenge response}
{
    name        = {challenge–response},
    text        = \mbox{\textit{challenge–response}},
    description = {Estrategia de los sistemas para la identificación de usuarios. Consiste en plantear un reto que requiere una determinada actuación voluntaria o involuntaria por parte del usuario.}
}

\newglossaryentry{CNN}
{
    name        = {CNN},
    description = {\textit{Convolutional Neural Networks}. Redes neuronales densas, inspiradas en la arquitectura perceptrón multicapa \cite{goodfellow2016deep}.}
}

\newglossaryentry{CMOS}
{
    name        = {CMOS},
    description = {\textit{complementary Metal-Oxide-Semiconductor}.Sensor compuesto de una matriz de cédulas fotoeléctricas que permiten registrar la imágenes en las cámaras digitales.}
}

\newglossaryentry{contactless}
{
    name        = {contactless},
    text        = {\textit{contactless}},
    description = {Paradigma \textit{Human-Machine Interface} (HMI) basado en una tecnología de sensores que no requieren contacto.}
}

\newglossaryentry{CSD}
{
    name        = {CSD},
    description = {\textit{Camera to Subject Distance}. Distancia entre la persona y el dispositivo de captura en una captura facial. \GLS{ICAO} Doc $9303$ \cite{doc20069303} fija valores mínimos para esta distancia en los documentos de viaje.}
}

\newglossaryentry{dactilar}
{
    name        = {biometría dactilar},
    text        = {dactilar},
    description = {Tipo de biométrica que usa como rasgo identificador las huellas dactilares.}
}

\newglossaryentry{D-EER}
{
    name        = D-EER,
    text        = {\mbox{D-EER}},
    description = {\textit{Detection Equal Error Rate}. Indica el valor en el falsas aceptaciones se igualan a los falsos rechazos. En el caso de PAD cuando el APCER se igual con BPCER.}
}

\newglossaryentry{de-morphing}
{
    name        = {de-morphing},
    text        = \mbox{\textit{<<de-morphing>>}},
    description = {Proceso de extracción de las identidades involucradas en el un \gls{morphing} facial.}
}

\newglossaryentry{Dermalog}
{
    name        = {Dermalog},
    description = {Empresa dedicada al desarrollo de \textit{software} de reconocimiento dactilar \cite{DermalogOnline}.}
}

\newglossaryentry{DET}
{
    name        = DET,
    description = {\textit{Error Trade-Off}. Gráfica que permite represar los ratios de error de un sistema.}
}

\newglossaryentry{Dlib}
{
    name        = Dlib,
    description = {Librería con algoritmos de \textit{machine learning} \cite{dlibOnline}.}
}

\newglossaryentry{DNI-e}
{
    name        = DNI-e,
    text        = \mbox{DNI-e},
    description = {Documento Nacional de Identidad electrónico. Documento de identidad en España. Cumple la normativa \gls{eID} de \GLS{ICAO} \cite{doc20069303}.}
}

\newglossaryentry{DMN}
{
    name        = {DMN},
    description = {\textit{De-morphing Network}. Arquitectura de \GLS{CNN} implementada y entrada para realizar el proceso de \gls{de-morphing} entre dos imágenes.}
}

\newglossaryentry{DVT}
{
    name        = {DVT},
    description = {\textit{Delaunay-Voronoi triangulation}. Algoritmo para la construcción de \gls{morphing} facial.}
}

\newglossaryentry{DPI}
{
    name        = {dpi},
    description = {\textit{Dot Per Pixel}. Medida de la resolución de una imagen que indica los píxeles por pulgada en un dispositivo o lo puntos por pulgada en una impresión.}
}

\newglossaryentry{EBTS}
{
    name        = {EBTS},
    description = {\textit{Electronic Biometric Transmission Specification}, Departamento del FBI \cite{FBIBioOnline} dedicada a los sistemas biometricos.}
} 

\newglossaryentry{EEN}
{
    name        = {EEN},
    description = {\textit{Enterprise Europe Network}, Red de empresas de asesoramiento a otras empresas de la \GLS{EU} \cite{EENOnline}.}
}

\newglossaryentry{eDL}
{
    name        = {eDL},
    description = {Base de datos consultada por lo sistemas \GLS{ABC} con información sobre le viajero.}
}

\newglossaryentry{EER}
{
    name        = {EER},
    description = {\textit{Equal Error Rate}. Indica el valor en el falsas aceptaciones \GLS{FAR} se iguala a los falsos rechazos \GLS{FRR}.}
}

\newglossaryentry{EES}
{
    name        = {EES},
    description = {\textit{Entry/Exit System}, Proceso de cruce de fronteras que se realiza en la \gls{e-gate}.}
}

\newglossaryentry{EIC}
{
    name        = {EIC},
    description = {\textit{International Electrotechnical Commission}. Organización europea para la definición de estándares.}
}

\newglossaryentry{eID}
{
    name        = {eID},
    description = {Estandar de ICAO \cite{ICAOOnline}, para los documentos de identidad \cite{doc20069303}. También se conoce así a una de las bases de datos que consultan los sistemas \GLS{ABC} con información de la identidad del viajero.}
}

\newglossaryentry{eigenfaces}
{
    name        = {eigenfaces},
    text        = \textit{eigenfaces},
    description = {Descriptor para la clasificación de caras. Expuesto en el estudio \cite{turk1991face}.}
}

\newglossaryentry{ePass}
{
    name        = {ePass},
    description = {Base de datos que consultan los sistemas \GLS{ABC} con información del viajero.}
}

\newglossaryentry{EPCER}
{
    name        = {EPCER},
    description = {\textit{Equal Presentation Classification Error Rate}. Tasa de error en la que coinciden los valores de \GLS{APCER} y de \GLS{BPCER}.}
}

\newglossaryentry{EPA}
{
    name        = {EPA},
    description = {\textit{Enrollment Presentation Attack} Ataques de presentación que se producen en la etapa de registro de un sistema \GLS{ABC} \textit{<<Segregated Two Step>>}.}
}

\newglossaryentry{EVPA}
{
    name        = {EVPA},
    description = {\textit{Enrolment \& Verification Presentation Attack} Ataques de presentación que se producen en la etapa de registro y en la etapa de verificación de un sistema \GLS{ABC} \textit{<<Segregated Two Step>>}.}
}

\newglossaryentry{embedding}
{
    name        = {embedding},
    plural      = {embeddings},
    text        = \textit{embedding},
    description = {Matriz generada por un codificador \GLS{CNN} y que contiene la información necesaria para la reconstrucción de la imagen codificada o para su verificación \cite{schroff2015facenet}.}
}

\newglossaryentry{eMRTD}
{
    name        = {eMRTD},
    description = {\textit{Electronic machine-readable travel document}. Documento legible por sistemas automáticos.}
}

\newglossaryentry{ergonomia}
{
    name        = {ergonomía},
    description = {Disciplina que analiza y optimiza la interacción humana con los elementos de un sistema.}
}

\newglossaryentry{eRP}
{
    name        = {eRP},
    description = {\textit{Electronic Residence Permit}. Documento electrónico que otorga el permiso de residencia en un país determinado. el formato de este tipo de documentos viene definido por \GLS{ICAO} \cite{ICAOOnline}}
}

\newglossaryentry{estereovision}
{
    name        = {estereovisión},
    description = {Construcción de información tridimensional de profundidad a partir de dos imágenes.}
}

\newglossaryentry{ETIAS}
{
    name        = {ETIAS},
    description = {\textit{European Travel Information and Authorisation System}. Sistema de pre-registro para \GLS{TCNVE} en el área \Gls{Schengen}. Aprobada por la en en $2018$ y comienza a ser operativa en $2022$ \cite{union2018directiveETIAS}.}
}

\newglossaryentry{EU}
{
    name        = {EU},
    description = {\textit{European Union}. Unión política y económica de $28$ países del continente europeo \cite{EUOnline}.}
}

\newglossaryentry{eVRC}
{
    name        = {eVRC},
    description = {\textit{Electronic Vehicle Registration Certificate}. Documento de registro de vehículos. Definido por \GLS{ICAO} \cite{ICAOOnline}.}
}

\newglossaryentry{eVISA}
{
    name        =eVISA,
    description = {\textit{Electronic Visa}. Visado electrónico. Documento definido por \GLS{ICAO} \cite{ICAOOnline}.}
}

\newglossaryentry{e-gate}
{
    name        = {e-gate},
    plural      = {e-gates},
    text        = {\mbox{\textit{e-gate}}},
    description = {Dispositivo en el que normalmente se realiza el proceso de validación (\GLS{EES}). Se caracteriza por tener una compuerta que impide el paso mientras no se produzca una identificación correcta.}
}

\newglossaryentry{e-kiosk}
{
    name        = {e-kiosk},
    text        = {\mbox{\textit{e-kiosk}}},
    description = {Dispositivo en el que normalmente se realiza el proceso de registro (\GLS{RTP}).}
}

\newglossaryentry{e-passport}
{
    name        = {e-passport},
    text        = {\mbox{\textit{e-passport}}},
    description = {Pasaporte electrónico que cumple las especificaciones de eMRTD fijadas por la ICAO.}
}

\newglossaryentry{FaceNet}
{
    name        = {FaceNet},
    description = {Reconocedor facial basado en CNN publicado en 2015 en el estudio \cite{schroff2015facenet}}
}

\newglossaryentry{facial}
{
    name        = {biometría facial},
    text        = {facial},
    description = {Tipo de biométrica que usa como rasgo identificador el rostro.}
}

\newglossaryentry{FADO}
{
    name        = {FADO},
    description = {\textit{False and Authentic Documents Online}. Base de datos con el formato de los pasaportes originales de cada país e información sobre tipos de falsificaciones en documentos de viaje.}
}

\newglossaryentry{FAR}
{
    name        = {FAR},
    description = {\textit{False Acceptance Ratio}. Tasa de errores de falsa aceptación.}
}

\newglossaryentry{FAST}
{
    name        = {FAST},
    description = {\textit{Features from Accelerated Segment Test}. Algoritmo de detección y descripción de de la características en una imagen. Descrito en el estudio  \cite{rosten2006machine}.}
}

\newglossaryentry{FastPass}
{
    name        = {FastPass},
    description = {Proyecto europeo FP$7$ dedicado a la investigación en el campo de los sistemas ABC \cite{FastPassOnline}.}
}

\newglossaryentry{FLIR}
{
    name        = {FLIR},
    description = {Empresa dedicada a la fabricación de sistemas infrarrojos, térmicos y de visión nocturna \cite{FLIROnline}.}
}

\newglossaryentry{FlyPAD}
{
    name        = {FlyPAD},
    description = {\textit{On the fly Presentation Attack Detecction}. Sistema de detección de ataques de presentación de forma dinámica.}
}

\newglossaryentry{FMR}
{
    name        = {FMR},
    description = {\textit{False Match Rate}. Error de falsa aceptación \GLS{FAR}.}
}

\newglossaryentry{FNMR}
{
    name        = {FNMR},
    description = {\textit{False Non-Match Rate}. Error de falso rechazo \GLS{FRR}.}
}

\newglossaryentry{FPR}
{
    name        = {FPR},
    description = {\textit{False Positive Rate}.}
}

\newglossaryentry{FP7}
{
    name        = {FP7},
    description = {Categoría de proyectos europeos adscritos al \textit{Seventh Framework Programme}.}
}

\newglossaryentry{FNR}
{
    name        = {TPR},
    description = {\textit{False Negative Rate}.}
}

\newglossaryentry{FRAV}
{
    name        = {FRAV},
    description = {\textit{Face Recognition and Artificial Vision}, Grupo de investigación de \GLS{URJC} \cite{FRAVOnline}.}
}

\newglossaryentry{FRAV-ABC}
{
    name        = {FRAV-ABC},
    text        = {\mbox{\textit{FRAV-ABC}}},
    description = {Base de datos con ataques creada por el grupo de investigación  \GLS{FRAV} de \GLS{URJC}.}
}

\newglossaryentry{FRAV-ABC-Attack}
{
    name        = {FRAV-ABC-Attack},
    text        = {\mbox{\textit{FRAV-ABC-Attack}}},
    description = {Base de datos con ataques creada por el grupo de investigación \GLS{FRAV} de \GLS{URJC} durante la implantación de los sistemas \GLS{ABC} del proyecto \GLS{ABC4EU}. Además de imágenes \gls{chip} de pasaportes incluye presentaciones \gls{vivo bona-fide} y ataques, en un sistemas \GLS{ABC} en un cruce de fronteras real.}
}

\newglossaryentry{FRAV-Morphing}
{
    name        = {FRAV-Morphing},
    text        = {\mbox{\textit{FRAV-Morphing}}},
    description = {Base de datos con ataques de \gls{morphing}, creada por el grupo de investigación \GLS{FRAV} de \GLS{URJC}. Está construida con  \gls{FRAV-ABC} y se divide en tren subconjuntos \gls{FRAV-Morphing-Test}, \gls{FRAV-Morphing-Val} y \gls{FRAV-Morphing-Train}.}
}

\newglossaryentry{FRAV-Morphing-Train}
{
    name        = {FRAV-Morphing-Train},
    text        = {\mbox{\textit{FRAV-Morphing-Train}}},
    description = {Subconjunto de datos de \textit{\gls{FRAV-Morphing}} para el entrenamiento de sistemas \GLS{MAD}.}
}

\newglossaryentry{FRAV-Morphing-Val}
{
    name        = {FRAV-Morphing-Val},
    text        = {\mbox{\textit{FRAV-Morphing-Val}}},
    description = {Subconjunto de datos de \textit{\gls{FRAV-Morphing}} para la validación de sistemas \GLS{MAD}.}
}

\newglossaryentry{FRAV-Morphing-Test}
{
    name        = {FRAV-Morphing-Test},
    text        = {\mbox{\textit{FRAV-Morphing-Test}}},
    description = {Subconjunto de datos de \textit{\gls{FRAV-Morphing}} para realizar test de sistemas \GLS{MAD}.}
}

\newglossaryentry{FRAV-Morphing-Test-PS-300}
{
    name        = {FRAV-Morphing-Test-P\&S-300},
    text        = {\mbox{\textit{FRAV-Morphing-Test-P}\&\textit{S-300}}},
    description = {Base de datos con ataques de \gls{morphing}, basada en \gls{FRAV-Morphing-Test} con sus imágenes impresas y escaneadas a $300$ \gls{DPI}, creada por el grupo de investigación \GLS{FRAV} de \GLS{URJC}.}
}

\newglossaryentry{FRAV-Morphing-Test-PS-150}
{
    name        = {FRAV-Morphing-Test-P\&S-150},
    text        = {\mbox{\textit{FRAV-Morphing-Test-P}\&\textit{S-150}}},
    description = {Base de datos con ataques de \gls{morphing}, basada en \gls{FRAV-Morphing-Test} con sus imágenes impresas y escaneadas a $150$ \gls{DPI}, creada por el grupo de investigación \GLS{FRAV} de \GLS{URJC}.}
}

\newglossaryentry{FRAV-OnTheFly}
{
    name        = {FRAV-OnTheFly},
    text        = {\mbox{\textit{FRAV-OnTheFly}}},
    description = {Base de datos multimodal de ataques de presentación en sistemas \GLS{ABC} \gls{OnTheFly} creada en laboratorio, por el grupo de investigación \GLS{FRAV} de \GLS{URJC}.}
}

\newglossaryentry{FRAV-ABC-OnTheFly}
{
    name        = {FRAV-ABC-OnTheFly},
    text        = {\mbox{\textit{FRAV-ABC-OnTheFly}}},
    description = {Base de datos multimodal de ataques de presentación en sistemas \GLS{ABC} \gls{OnTheFly} creada en un sistema \GLS{ABC} real en el puerto de Algeciras (España), por el grupo de investigación \GLS{FRAV} de \GLS{URJC}.}
}

\newglossaryentry{FRAV-Attack}
{
    name        = {FRAV-Attack},
    text        = {\mbox{\textit{FRAV-Attack}}},
    description = {Base de datos multimodal de ataques de presentación creada por el grupo de investigación \GLS{FRAV} de \GLS{URJC}.}
}

\newglossaryentry{frontex}
{
    name        = {frontex},
    description = {Agencia Europea de la Guardia de Fronteras y Costas \cite{FRONTEXOnLine}}
}

\newglossaryentry{FRR}
{
    name        = FRR,
    description = {\textit{False rejections Rate}. Tasa de errores de falso rechazo,}
}

\newglossaryentry{FS-SPN}
{
    name        = {FS-SPN},
    text        = \mbox{FS-SPN},
    description = {\textit{Fourier Spectrum of Sensor Pattern Noise}. Evaluación del ruido de una imagen analizando su espectro de Furier. Descrito en el estudio \cite{zhang2018face}.}
}

\newglossaryentry{FTA}
{
    name        = {FTA},
    description = {\textit{Failure to Adquire}. Fallo en el subsistema de captura (\GLS{FTD} o \GLS{FTC}) o en la extracción de caracteristicas (\GLS{FTP}) en un sistema biométrico .}
}

\newglossaryentry{FTC}
{
    name        = {FTC},
    description = {\textit{Failure to Capture}. Fallo en el subsistema de captura de un sistema biométrico.}
}

\newglossaryentry{FTD}
{
    name        = {FTD},
    description = {\textit{Failure to Detect}. Fallo en la detección del subsistema de captura de un sistema biométrico.}
}

\newglossaryentry{FTE}
{
    name        = {FTE},
    description = {\textit{Failure to Enroll}. Fallo al construir la plantilla de referencia de un usuario al registrarlo en un sistema biométrico.}
}

\newglossaryentry{FTIR}
{
    name        = {FTIR},
    description = {\textit{Frustrated Total Internal Reflexion}. Tipo de lectores de huellas de tipo óptico, basados en la reflexión de la luz a través un prisma.}
}

\newglossaryentry{FTP}
{
    name        = {FTP},
    description = {\textit{Failure to Process}. Fallo en la extracción de las características de un rasgo en un sistema biométrico.}
}
   
\newglossaryentry{GAP}
{
    name        = {GAP},
    description = {\textit{Pluging} para \GLS{Gimp} para la construcción de \gls{morphing} facial.}
}

\newglossaryentry{genuino}
{
    name        = {genuino},
    description = {Viajero que es el propietario legítimo de la documento que porta.}
}

\newglossaryentry{GAN}
{
    name        = {GAN},
    description = {\textit{Generative Adversarial Networks}.}
}

\newglossaryentry{Gemalto}
{
    name        = {Gemalto},
    description = {Empresa dedicada al desarrollo de sistemas ABC.}
}

\newglossaryentry{Gimp}
{
    name        = {Gimp},
    description = {Programa para de diseño gráfico con un \textit{pluging} GAP que permite hacer \gls{morphing} faciales}
}

\newglossaryentry{HOG}
{
    name        = {HOG},
    description = {\textit{Histogram of Oriented Gradients}. Descriptor de imágenes presentado en \cite{mcconnell1986method} y usado por primera vez en biometría facial en el articulo \cite{shu2011histogram}}
}

\newglossaryentry{HMI}
{
    name        = {HMI},
    description = {\textit{Human-Machine Interface}.Parte del software o del hardware encargado de la comunicación entre el sistema y el usuario.}
}

\newglossaryentry{HMM}
{
    name        = {HMM},
    description = {\textit{Hidden Markov Model}. Modelos ocultos de Markov. Modelo estadístico para el reconocimiento de patrones.}
}

\newglossaryentry{IATA}
{
    name        = {IATA},
    description = {\textit{International Air Transport Association}. Asociación de lineas aéreas internacionales que representan el 82\% del trafico mundial.}
}

\newglossaryentry{ICAO}
{
    name        = {ICAO},
    description = {\textit{International Civil Aviation Organization}. También conocida como OACI, Organización de Aviación Civil Internacional \cite{ICAOOnline}}
}

\newglossaryentry{IEC}
{
    name        = {IEC},
    description = {\textit{International Electrotechnical Commission}. Organización internacional para el normalizado y la estandarización en sistemas electrónicos e informáticos \cite{IECOnline}.}
}

\newglossaryentry{IED}
{
    name        = {IED},
    description = {\textit{Inter Eye Distance}. Distancia entre el centro de cada ojo en una captura de biometría facial. \GLS{ICAO} Doc $9303$ \cite{doc20069303} fija valores mínimos para esta distancia en los documentos de viaje.}
}

\newglossaryentry{INDRA}
{
    name        = {INDRA},
    text        = \mbox{INDRA},
    description = {INDRA. Empresa dedicada al desarrollo de sistemas ABC.}
}

\newglossaryentry{Innovative}
{
    name        = {Innovative},
    text        = \mbox{Innovative},
    description = {Empresa dedicada al desarrollo de sistemas ABC.}
}

\newglossaryentry{INTERPOL}
{
    name        = {INTERPOL},
    text        = \mbox{INTERPOL},
    description = {\textit{International Criminal Police Organization} \cite{INTERPOLOnline}.}
}

\newglossaryentry{impostor}
{
    name        = {impostor},
    description = {Viajero que se está haciendo pasar por otro, suplantando su identidad.}
}

\newglossaryentry{IOM}
{
    name        = {IOM},
    description = {\textit{Iris on the Move} Biometría del \gls{iris} con adquisición no estática. Presentada en el el estudio \cite{matey2006iris}.}
}

\newglossaryentry{IR}
{
    name        = {IR},
    description = {Radiación infrarroja, Radiación electromagnética con longitud de onda mayor que la de la luz visible y frecuencia menor.}
}

\newglossaryentry{iris}
{
    name        = {biometría iris},
    text        = {iris},
    description = {Membrana del ojo con un patrón único en cada individuo que puede usarse como rasgo identificativo en biometría. Es uno de los rasgos biométricos recomendados por \GLS{ICAO} para la identificación de pasaportes en sistemas \GLS{ABC} \cite{doc20069303}.}
}

\newglossaryentry{ISO}
{
    name        = {ISO},
    description = {\textit{International Organization for Standardization}.}
}

\newglossaryentry{JPEG}
{
    name        = {JPEG},
    description = {\textit{Joint Photographic Experts Group}. Formato de compresión y de codificación de imágenes.}
}

\newglossaryentry{JPG2000}
{
    name        = {JPG2000},
    description = {\textit{Joint Photographic Experts Group}. Formato de compresión y de codificación de imágenes recomendados en los estándares \GLS{ISO}/\GLS{IEC} $19794$ \cite{ISO/Format} para los sistemas biométricos.}
}

\newglossaryentry{K-NN}
{
    name        = {K-NN},
    text        = {\mbox{K-NN}},
    description = {\textit{k-nearest neighbors} Clasificador supervisado basado en la distancia en entre muestras.}
}

\newglossaryentry{landmark}
{
    name        = {landmark},
    plural      = {landmarks},
    description = {Puntos de refería. Los \textit{landmark} faciales son un conjunto de puntos característicos de una imagen facial (iris, cejas, labios etc.)}
}
 
\newglossaryentry{LBP}
{
    name        = {LBP},
    description = {\textit{Local Binary Pattern}. Descriptor de imágenes, descrito en la publicación \cite{ojala2000gray}.}
}

\newglossaryentry{LFW}
{
    name        = {LFW},
    description = {\textit{Labeled Faces in the Wild}. Base de datos para estudios de biometría facial \cite{huang2008labeled}.}
}

\newglossaryentry{liveness}
{
    name        = {liveness},
    text        = {\textit{liveness}},
    description = {Características o cualidades que indican la viveza en la presentación a un sistema biométrico.}
}

\newglossaryentry{liveness detection}
{
    name        = {liveness detection},
    text        = {\mbox{\textit{liveness detection}}},
    description = {Algoritmos que tratan de detectar \gls{liveness} en la adquisición biométrica.}
}

\newglossaryentry{LSTM}
{
    name        = {LSTM},
    description = {\textit{Long short-term memory}. Redes neuronales de aprendizaje profundo cuyas conexiones tienen retroalimentación. Suele usarse con para procesar datos secuenciales, como vídeos.}
}

\newglossaryentry{nacional-ID}
{
    name        = {nacional-ID},
    text        = {\mbox{nacional-ID}},
    description = {Documento nacional de identidad de cada país. Algunos \GLS{ABC} son capaces de procesar este tipo de documentos si cumplen la normativa de \GLS{ICAO} \cite{doc20069303} para este tipo de documentos.}
}

\newglossaryentry{NFC}
{
    name        = {NFC},
    description = {\textit{Near Field Communication}. Tecnología que permite la comunicación entre dispositivos cercanos ($\pm$10cm).}
}

\newglossaryentry{NFIQ}
{
    name        = {NFIQ},
    description = {\textit{NIST Fingerprint Image Quality}. Algoritmo estándar desarrollado por \GLS{NIST} para evaluar la calidad de las imágenes de huellas dactilar.}
}

\newglossaryentry{NIST}
{
    name        = {NIST},
    description = {\textit{National Institute of Standards and Technology}.Agencia estadounidense encargada de la creación de estándares tecnológicos.}
}

\newglossaryentry{NN}
{
    name        = {NN},
    description = {\textit{Nearest Neighbour Classifier}. Clasificador basado en la regla del vecino más cercano.}
}

\newglossaryentry{MAD}
{
    name        = {MAD},
    plural      = {MADs},
    description = {\textit{Morphing Attack Detection}. Métodos \GLS{PAD} para ataques \gls{morphing}.}
}

\newglossaryentry{MAD sin referencia}
{
    name        = {MAD sin referencia},
    text        = {\mbox{MAD sin referencia}},
    description = {\textit{Nor Reference Morphing Attack Detection}. Métodos de \GLS{MAD} cuando se dispone de una única imagen, la imagen posiblemente alterada.}
}

\newglossaryentry{MAD diferencial}
{
    name        = {MAD Diferencial},
    text        = {\mbox{MAD Diferencial}},
    plural      = {MAD Diferenciales},
    description = {\textit{Diferential Morphing Attack Detection}. Métodos de \GLS{MAD} cuando se dispone de dos imágenes: El posible \gls{morphing} y una imagen de alguna de la identidades combinadas.}
}

\newglossaryentry{MSE}
{
    name        = {MSE},
    description = {\textit{Mean Squared Error}. Estimación del error que se usa para los entrenamientos de \GLS{CNN}.}
}

\newglossaryentry{Replay Mobile}
{
    name        = {Replay Mobile},
    text        = {\mbox{\textit{Replay Mobile}}},
    description = {Base de datos de ataques de presentación realizados con la pantalla de un dispositivo móvil.}
}

\newglossaryentry{mantrap}
{
    name        = {mantrap},
    text        = {\textit{mantrap}},
    description = {Dispositivo de lo sistemas ABC que por seguridad retine al pasajero durante la identificación.}
}

\newglossaryentry{micro-texturas}
{
    name        = {micro-texturas},
    text        = {\mbox{micro-texturas}},
    description = {Detalles no perceptibles de la imágenes, con información descriptiva que puede ser extraída mediante métodos como \GLS{LBP} \cite{ojala2000gray}.}
}

\newglossaryentry{MobilePass}
{
    name        = {MobilePass},
    description = {Proyecto europeo FP$7$ para el desarrollo de sistemas ABC móviles.}
}

\newglossaryentry{MorphDB}
{
    name        = {MorphDB},
    description = {Base de datos de ataques de \gls{morphing} \cite{ferrara2017face}.}
}

\newglossaryentry{Modentity}
{
    name        = {Modentity},
    text        = {\mbox{Modentity}},
    description = {Proyecto de KIRAS (\textit{Austrian Security Research Programme}) dedicado a la investigación de sistemas biométricos móviles.}
}

\newglossaryentry{MODI}
{
    name        = {MODI},
    description = {\textit{MODI-Vision for identification}. Empresa dedicada al desarrollo de sistemas ABC.}
}

\newglossaryentry{mono-modal}
{
    name        = {mono-modal},
    plural      = {mono-modales},
    text        = \mbox{mono-modal},
    description = {Sistemas biométricos que usan un sólo rasgo biométrico para realizar la identificación.}
}

\newglossaryentry{MorGAN}
{
    name        = {MorGAN},
    description = {Base de datos con imágenes \gls{morphing} facial construida con redes convolucionales \GLS{GAN} \cite{damer2018morgan}.}
}

\newglossaryentry{morphing}
{
    name        = {morphing},
    text        = {\mbox{\textit{morphing}}},
    description = {Proceso de fusión de dos o más imágenes en una. Puede usarse como ataque de presentación que consiste en la fusión de varias identidades en una única imagen.}
}

\newglossaryentry{MMPMR}
{
    name        = {MMPMR},
    description = {\textit{Mated Morph PresentationMatch Rate}. Ratio propuesta para medir la vulnerabilidad de un sistema biométrico facial a los ataques de \gls{morphing} \cite{scherhag2017biometric}.}
}

\newglossaryentry{MRTD}
{
    name        = {MRTD},
    description = {\textit{Machine Readable Travel Document}. Maquinas capaces de leer pasaportes y otros documentos de viaje que cumplan los estándares fijados por ICAO Doc $9303$ (avalados por ISO y por IEC (ISO/IEC $7501$-$1$))}
}

\newglossaryentry{MRZ}
{
    name        = {MRZ},
    description = {\textit{Machine Readable Zone}. Región de los \gls{eMRTD} con información del viajero en formato texto que puede ser leída mediante \GLS{OCR}. Definido en ICAO Doc. $9303$ \cite{doc20069303} y avalado por ISO y por IEC (ISO/IEC $7501$-$1$) \cite{ISO/PADFramework}}
}

\newglossaryentry{MSU MFSD}
{
    name        = {MSU MFSD},
    text        = {\mbox{MSU MFSD}},
    description = {Base de datos de ataques de presentación.}
}

\newglossaryentry{multi-biometria}
{
    name        = {multi-biometría},
    plural      = {multi-biometrías},
    text        = {\textit{multi-biometría}},
    description = {Uso de distintos rasgos biométricos, diferentes capturas o distintos clasificadores para la identificación de usuarios.}
}

\newglossaryentry{multi-modal}
{
    name        = {multi-modal},
    plural      = {multi-modales},
    text        = {\mbox{multi-modales}},
    description = {Uso de distintos rasgos biométricos para la identificación de usuarios.}
}

\newglossaryentry{OCR}
{
    name        = {OCR},
    description = {\textit{Optical Character Recognition}. Reconocimiento de caracteres para la digitalización de textos.}
}

\newglossaryentry{ONU}
{
    name        = {ONU},
    description = {Organización de las Naciones Unidas. Organización que trata de  prevenir conflictos y poner de acuerdo a las partes implicadas \cite{ONUOnline}.}
}

\newglossaryentry{OnTheFly}
{
    name        = {OnTheFly},
    text        = {\textit{OnTheFly}},
    description = {Tipo de captura de objeto o individuos en movimiento. Los sistemas \GLS{ABC} con este tipo de captura se conoce como \GLS{ABC} \textit{OnTheFly}.}
}

\newglossaryentry{OpenCV}
{
    name        = {OpenCV},
    description = {Biblioteca de \textit{software} libre especializada en visión artificial \cite{openCVOnline}.}
}

\newglossaryentry{ORB}
{
    name        = {ORB},
    description = {\textit{Oriented fast and Rotated BRIEF}. Algoritmo de detección y de descripción de  características.}
}

\newglossaryentry{PA}
{
    name        = {PA},
    description = {\textit{Presentation Attack}. Ataque al subsistema de captura de sistema biométrico que consiste en la suplantación de alguno de los usuarios del sistema. También conocido como \gls{spoofing}.}
}

\newglossaryentry{PAI}
{
    name        = {PAI},
    plural      = {PAIs},
    description = {\textit{Presentation Attack Instrument}. Rasgo biométrico u objeto empleado para realizar un ataque de presentación.}
}

\newglossaryentry{PAD}
{
    name        = {PAD},
    description = {\textit{Presentation Attack Detection}. Sistema automático que detecta los ataques de presentación}
}

\newglossaryentry{Personalausweis}
{
    name        = {Personalausweis},
    text        = {\mbox{\textit{Personalausweis}}},
    description = {Documento de identidad en Alemania. Cumple la normativa \gls{eID} de ICAO \cite{doc20069303}.}
}

\newglossaryentry{PIDaaS}
{
    name        = {PIDaaS},
    description ={\textit{Private Identity as a Service}. Proyecto europeo del 7º Programa Marco de Innovación y Competitividad (CIP).}
}

\newglossaryentry{PKD}
{
    name        = {PKD},
    description = {\textit{Public Key Directory}. Base de datos de \GLS{ICAO} con las claves públicas necesarias para validar y autentificar los pasaportes electrónicos.}
}

\newglossaryentry{PKI}
{
    name        = {PKI},
    description = {\textit{Public Key Infraestructure}.Conjunto de tecnologías para sistemas de cifrado y firma digital.}
}

\newglossaryentry{PNG}
{
    name        = {PNG},
    description = {\textit{Portable Network Graphics}. Formato de compresión y de codificación de imágenes recomendados en los estándares \GLS{ISO}/\GLS{IEC} $19794$ \cite{ISO/Format} para los sistemas biométricos.}
}

\newglossaryentry{PMDB}
{
    name        = {PMDB},
    description = {Base de datos sobre ataques de \gls{morphing} \cite{ferrara2017face}.}
}

\newglossaryentry{PRNU}
{
    name        = {PRNU},
    description = {\textit{Photo Response Non-Uniformity}. Medida de calidad de las imágenes atendiendo a factores de luminosidad y ruido. Propuesto por primera vez en \cite{lukas2006digital} para detectar el dispositivo origen de una imagen por su patrón de ruido.}
}

\newglossaryentry{Replay Attack}
{
    name        = {Replay Attack},
    text        = {\mbox{\textit{Replay Attack}}},
    description = {Base de datos publica con ataques de presentación.}
}

\newglossaryentry{RFID}
{
    name        = {RFID},
    description = {\textit{Radio Frequency Identification}. Chip que almacena información para la identificación automática de objetos o personas. Los \gls{eMRTD} incluyen este tipo de chip con los datos del viajero.}
}

\newglossaryentry{RGB}
{
    name        = {RGB},
    description = {Modelo de color basado en la intensidad de tres colores primarios rojo el verde y el azul.}
}

\newglossaryentry{ROC}
{
    name        = {ROC},
    description = {\textit{Relative Operating Characteristic}. Gráfica que permite visualizar y comparar el rendimiento de un clasificador binario.}
}

\newglossaryentry{RMMR}
{
    name        = {MMPMR},
    description = {\textit{Relative Morph Match Rate}. Ratio propuesta para medir la vulnerabilidad de un sistema biométrico facaial a los ataques de \gls{morphing} \cite{scherhag2017biometric}.}
}

\newglossaryentry{RTP}
{
    name        = {RTP},
    description = {\textit{Register Traveller Programme}, Proceso de registro del viajero.}
}

\newglossaryentry{SBC}
{
    name        = {SBC},
    description = {\textit{Schengen Border Code}, Código que establece las normas a cumplir dentro de las fronteras de la zona \Gls{Schengen} \cite{SBCode2016}.}
}
 
\newglossaryentry{Schengen}
{
    name        = {Schengen},
    text        = {\mbox{\textit{Schengen}}},
    description = {Acuerdo que elimina los controles de frontera entre algunos países europeos y no europeos.}
}

\newglossaryentry{SIFT}
{
    name        = SIFT,
    description = {\textit{Scale Invariant Feature Transform}. Descriptor de imágenes presentado en el articulo \cite{lowe2004distinctive}.}
}

\newglossaryentry{SIS}
{
    name        = SIS,
    description = {Sistema de Información de \gls{Schengen}. Base de datos con información de personas y de objetos para controlar y garantizar la seguridad del espacio \gls{Schengen}. (Actualmente SIS II)}
}

\newglossaryentry{spoofing}
{
    name        = soopfing,
    description = {Ataque al subsistema de captura de sistema biométrico que consiste en la suplantación de alguno de los usuarios del sistema. También conocido como \textit{Presentation Attack} (\GLS{PA}).}
}

\newglossaryentry{SURF}
{
    name        = SURF,
    description = {\textit{Speeded Up Robust Features}. Descriptor de imágenes presentado en el articulo \cite{bay2008speeded}.}
}

\newglossaryentry{SVM}
{
    name        = SVM,
    plural      = SVMs,
    description = {\textit{Support Vector Machine} Algoritmo de aprendizaje supervisado.}
}

\newglossaryentry{TCN}
{
    name        = TCN,
    description = {\textit{Third Country Nationals}. Personas cuya nacionalidad requiere visados para otros países. Los viajeros del área \Gls{Schengen} con nacionalidades de países no miembros son TCN.}
}

\newglossaryentry{TCNVE}
{
    name        = TCNVE,
    description = {\textit{Third Country Nationals Visa-exempt}. Viajeros \GLS{TCN} que no requieren de visado ni de permiso de residencia para acceder a los países miembros del área \Gls{Schengen}.}
}

\newglossaryentry{TCNVH}
{
    name        = TCNVH,
    description = {\textit{Third Country Nationals Visa Holders}. Viajeros \GLS{TCN} que requieren de un visado o de permiso de residencia para acceder a los países miembros del área \Gls{Schengen}.}
}

\newglossaryentry{TDA}
{
    name        = TDA,
    description = {\textit{Topological Data Analysis}. Analiza las micro-texturas de una imagen. Con imágenes suele implementarse mediante descriptores \GLS{LBP}.}
}

\newglossaryentry{token}
{
    name        = token,
    text        = \textit{token},
    description = {Elemento físico o lógico que algunos sistemas \GLS{ABC} requieren para la identificación del los viajeros.}
}

\newglossaryentry{tokenless}
{
    name        = tokenless,
    text        = \mbox{\textit{tokenless}},
    description = {Tipo de sistemas \GLS{ABC} que no requieren de un \gls{token} para la identificación del los viajeros.}
}

\newglossaryentry{token-required}
{
    name        = token-required,
    text        = {\mbox{\textit{token-required}}},
    description = {Tipo de sistemas \GLS{ABC} que requieren de un \gls{token}, físico o lógico, para la identificación del los viajeros.}
}

\newglossaryentry{touchless}
{
    name        = touchless,
    text        = {\textit{touchless}},
    description ={Paradigma \textit{Human-Machine Interface} (HMI) basado en una tecnología de sensores que no requieren contacto.}
}

\newglossaryentry{TPR}
{
    name        = TPR,
    description = {\textit{True Positive Rate}.}
}

\newglossaryentry{URJC}
{
    name        = {URJC},
    description = {Universidad Rey Juan Carlos \cite{urjcOnline}.}
}

\newglossaryentry{usabilidad}
{
    name        = {usabilidad},
    description = {Disciplina que analiza y optimiza los sistemas de forma que sean fáciles de utilizar por los usuarios.}
}

\newglossaryentry{UV}
{
    name        = {UV},
    description = {Ultravioleta. Intervalo de las frecuencias de onda de la luz que no es visible por el ojo humano. }
}

\newglossaryentry{VGG19}
{
    name        = VGG19,
    description = {Arquitectura CNN para visión artificial presentada en la publicación \cite{simonyan2014very}.}
}

\newglossaryentry{VGG-Face}
{
    name        = VGG-Face,
    text        = {\mbox{VGG-Face}},
    description = {Arquitectura CNN bassada en VGG19 especializada en el reconocimiento facial, presentada en la publicación \cite{parkhi2015deep}.}
}

\newglossaryentry{Viola-Jones}
{
    name        = Viola-Jones,
    text        = {\mbox{\textit{Viola-Jones}}},
    description ={Algoritmo para la detección habitualmente empleado para la detección de caras en imágenes \cite{viola2004robust}.}
}

\newglossaryentry{vivo}
{
    name        = {vivo},
    text        = {\textit{<<vivo>>}},
    description = {Información biométrica del viajero capturada directamente por los dispositivos \GLS{ABC}. La presentación puede ser un \gls{vivo bona-fide} o un \gls{vivo ataque}.}
}

\newglossaryentry{vivo bona-fide}
{
    name        = {vivo bona-fide},
    text        = {\textit{<<vivo \mbox{bona-fide}>>}},
    description = {Información biométrica del viajero capturada directamente por los dispositivos \GLS{ABC} cuando se realiza una presentación \gls{bona-fide}.}
}

\newglossaryentry{vivo ataque}
{
    name        = {vivo ataque},
    text        = {\textit{<<vivo ataque>>}},
    description = {Información biométrica del viajero capturada directamente por los dispositivos \GLS{ABC} cuando se realiza un presentación de ataque con algún \GLS{PAI}.}
}

\newglossaryentry{vivo registro}
{
    name        = {vivo registro},
    text        = {\textit{<<vivo registro>>}},
    description = {Información biométrica del viajero capturada directamente por los dispositivos \GLS{ABC} en la etapa \GLS{RTP}.}
}

\newglossaryentry{vivo registro bona-fide}
{
    name        = {vivo registro bona-fide},
    text        = {\textit{<<vivo registro \mbox{bona-fide}>>}},
    description = {Información biométrica del viajero capturada directamente por los dispositivos \GLS{ABC} en la etapa \GLS{RTP} con una presentación \gls{bona-fide}.}
}

\newglossaryentry{vivo registro ataque}
{
    name        = {vivo registro ataque},
    text        = {\textit{<<vivo registro ataque>>}},
    description = {Información biométrica del viajero capturada directamente por los dispositivos \GLS{ABC} en la etapa \GLS{RTP} con una presentación de ataque.}
}

\newglossaryentry{vivo validacion}
{
    name        = {vivo validación},
    text        = {\textit{<<vivo validación>>}},
    description = {Información biométrica del viajero capturada directamente por los dispositivos \GLS{ABC} en la etapa \GLS{EES}.}
}

\newglossaryentry{vivo validacion bona-fide}
{
    name        = {vivo registro bona-fide},
    text        = {\textit{<<vivo registro \mbox{bona-fide}>>}},
    description = {Información biométrica del viajero capturada directamente por los dispositivos \GLS{ABC} en la etapa \GLS{EES} con una presentación de \gls{bona-fide}.}
}

\newglossaryentry{vivo validacion ataque}
{
    name        = {vivo registro ataque},
    text        = {\textit{<<vivo registro ataque>>}},
    description = {Información biométrica del viajero capturada directamente por los dispositivos \GLS{ABC} en la etapa \GLS{EES} con una presentación de ataque.}
}

\newglossaryentry{VIS}
{
    name        = VIS,
    description = {\textit{Visa Information System}. Base de datos con la información de visados de corta duración entre los Estados Schengen.}
}

\newglossaryentry{visa}
{
    name        = {visa},
    description = {Permiso de estancia temporal en un país.}
}

\newglossaryentry{VisionBox}
{
    name        = {VisionBox},
    text        = {\textit{VisionBox}},
    description = {Empresa dedicada al desarrollo de sistemas ABC.}
}

\newglossaryentry{VIZ}
{
    name        = {VIZ},
    description = {\textit{Visual Inspection Zone}. Área preparada para una inspección visual dentro la hoja de datos en los documentos \gls{eMRTD} definidos por \gls{ICAO} en \cite{doc20069303}.}
}

\newglossaryentry{VPA}
{
    name        = {VPA},
    description = {\textit{Verification Presentation Attack} Ataques de presentación que se producen en la etapa de verificación de un sistema \GLS{ABC} \textit{<<Segregated Two Step>>}.}
}

\newglossaryentry{VRC}
{
    name        = {VRC},
    description = {\textit{Vehicle Registration Certificate}. Base de datos con información e registro de vehículos, usada por algunos sistemas controles de frontera.}
}

\newglossaryentry{WLMP}
{
    name        = {WLMP},
    description = {\textit{Weighted Local Magnitude Patterns}. Descriptor de texturas de la imagen, similar \GLS{LBP} pero sin binarizar el descriptor del píxel.}
}

\newglossaryentry{YCbCr}
{
    name        = {YCbCr},
    description = {Espacio de color para sistemas de imagen digital, basada en las diferencias de cromatancia del rojo, del azul y del amarillo.}
}

\newglossaryentry{YTF}
{
    name        = {YTF},
    description = {\textit{YouTube Faces}. Base de datos para estudios de biometría facial con caras extraídas de \textit{YouTube} \cite{wolf2011face}.}
}

\newglossaryentry{ZeroFAR}
{
    name        = {ZeroFAR},
    description = {Valor del error de falso rechazo (\GLS{FRR}) cuando el error de falsa aceptación (\GLS{FAR}) es $0$.}
}

\newglossaryentry{ZeroFRR}
{
    name        = {ZeroFRR},
    description = {Valor del error de falsa aceptación (\GLS{FAR}) cuando el error de falso rechazo (\GLS{FRR}) es $0$.}
}
